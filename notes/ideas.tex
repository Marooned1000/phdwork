\documentclass{article}

\usepackage{amsfonts}
\usepackage{amssymb}

\begin{document}

\section{Game Thoery}

\subsection{Weighted Voting Games (WVGs)}

\cite{Chalkiadakis_simplecoalitional} A WVG is described by its set of players $N$, a vector of players' $weights$ w = (w_1,...,w_n), $w_i \in \mathbb{R}$ for $i \in N$, and a $threshold$ $q \in \mathbb{R}$ we write $G = (N; w; q)$. The utility function $u(S)$ of a game $G = (N; w; q)$ is given by $u(S) = 1$ iff $\sum_{i \in S} w_i \geq q.$

In some cases, the only acceptable outcome of a game is the formation of the $grand coalition$, i.e., the coalition of all players $N$. However, in many multi-agent scenarios it is more natural for agents to split into groups so that each group performs its own task. This is captured by the notion of a $coalition structure$, which is a prtition of the set of agents $N$.

\subsubsection{The Idea}
The idea here can be make the utility function 0 or 1 based on a threshold where that amount of web services or agents can perform a task. Like instead of having real utilities we say agents 1,2,3 can do the task we can make it function of their QoS parameters.



\bibliographystyle{plain}
\bibliography{ehsan}

\end{document}
\documentclass{article}

\usepackage{amsfonts}
\usepackage{amssymb}

\begin{document}

\section{General}

In statistics, "empirical" quantities are those computed from observed values, as opposed to derived from theoretical considerations.

\hrule 

\cite{koch2010introduction}If the statements $A_1,A_2,...A_n$ are not only mutually exclusive but also $exhustive$ which means that the background information $C$ stipulates that one and only one statement must be true and if one is true the remaining statements must be false, then we obtain with (2.13) and (2.15) from (2.21)
\begin{equation}\label{eq:exhustive}
P(A_1 + A_2 + ... + A_n | C) = \sum_{i=1}^{n}P(A_i|C) = 1.
\end{equation}

\hrule



\bibliographystyle{plain}
\bibliography{prob}

\end{document}
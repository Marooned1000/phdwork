\documentclass{article}

\usepackage{amsfonts}
\usepackage{amssymb}

\begin{document}

\section{Computation Complexity and Intractability}

\subsection{NP-Completeness}

To prove that $\prod$ is NP-complete, we merely show that \cite{Garey:1990:CIG:574848}

\begin{enumerate}
\item $\prod \in$ NP and,
\item some known NP-complete problem $\prod^\prime$ transforms to $\prod$.
\end{enumerate}

\section{Game Theory}

\subsection{Learning in Games}
Stochastic games generalize both Markov decision processes and repeated games.

\section{Paper Reviewers}
I strongly believe that research in this direction is very important as service computing get matured and move towards seamless integration of web services dynamically.

This paper proposes an efficient coalition formation mechanism using cooperative game-theoretic techniques. The mechanism could be useful for web services to form stable groups allowing them to maximize their efficiency and generate near-optimal communities.

The related work and introduction are in Section 1. The authors only give a review and analyze of references [6]~[9],which is not adequate. I also found some other references in the Performance Formulation and Modeling Section, but it is required to summary them is the related work section to give a clear comparison.


\section{To add on proposal}

\subsection{a good intro}
Coalition-game thoery provides suitable analytical tools that have been widely explored in different diciplines such as economics and political sceince. With the recent emergence of cooperation as a new networking paradigm, coalition-game  theory started to become a central framework for modeling cooperation in wireless and communication networks. For instance, coalition games prove to be a very powerful tool for designing fair, robust, practical and efficient cooperation strategies in communication networks. 

\subsection{contractnet}

Using round robin instead of contractnet, to reduce the simulation complexity. Both have identical behavoiurs.
we dont want WS not bid if they are not working at that moment.

\subsection{other}

adding "`towards defingin and assessing the non-functional properties of communities of web services"' as good refrence for QoS metrics.



adding a tree chart of what we cover in proposed method section.

adding shapely value properties.

\bibliographystyle{plain}
\bibliography{ehsan}

\end{document}
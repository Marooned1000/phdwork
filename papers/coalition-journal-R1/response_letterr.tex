
%\documentclass[preprint,authoryear,12pt]{elsarticle}
\documentclass[times, 12pt,a4paper]{article}
\usepackage{graphicx,amssymb,amsthm,stmaryrd}%,booktabs}
\usepackage{times}
\usepackage{amsmath,url,algorithm,algorithmic,mathrsfs,makeidx}
\usepackage{graphicx}
\usepackage{epstopdf}
\usepackage{epsfig}
\usepackage{latex8}

%journal{Journal of Knowledge-Based Systems}



\begin{document}
%\begin{frontmatter}

\newtheorem{definition}{Definition}
\newtheorem{proposition}{Proposition}
\newtheorem{example}{Example}
\newtheorem{lemma}{Lemma}
\newtheorem{theorem}{Theorem}
\newtheorem{corollary} {Corollary}
\title{\underline{Response Letter}\\\vspace{0.5cm}Efficient Community Formation for Web Services}
\author{Ehsan Khosrowshahi Asl\\
e\_khosr@encs.concordia.ca\\
\and
Hadi Otrok\\
hadi.otrok@kustar.ac.ae \\
\and
Jamal Bentahar\\
bentahar@ciise.concordia.ca \\
\and
Rabeb Mizouni\\
rabeb.mizouni@kustar.ac.ae \\
}


\maketitle








First, we would like to thank and express our appreciation to the
reviewers for taking the time to carefully review our manuscript,
and for their insightful, valuable, and very useful comments. We
would like to thank the editor as well for his valuable comments
and suggestions for improvements. We carefully considered the
comments and implemented all of them in the revised version. In
this letter, we will explain, point by point, how the issues
raised in the reviews are addressed and answered. We start by
reporting a summary of changes.

\begin{center}
\textbf{\textit{ Summary of changes:}}
\end{center}






\begin{itemize}
\item  We fixed the paradoxes highlighted in the previous version
of the manuscript by proposing CTLKC$^+$, a new logic that can be
used to reason simultaneously about knowledge and social
commitments in multi-agent systems.

\item  We added the model, syntax and semantics of CTLKC$^+$.

\item  We explained the new social accessibility relation of
CTLKC$^+$, its intuition and its properties.

             %\item  We fixed eight paradoxes using the semantics of CTLKC$^+$ logic by providing formal proofs for the valid properties and models (counterexamples) for the non-valid properties.
             %\item  We removed paradox P7 (the reasons for that will be explained later).
             %\item  We added some references recommended by the reviewers.

\item We outlined a model checking technique for CTLKC$^+$ based
on the reduction from the model checking problem of GCTL$^*$, a
generalized logic of CTL*.

\item We analyzed the computational complexity of the model
checking problem of CTLKC$^+$ and proved that this complexity is
the same as the one of model checking CTLK and CTLC separately.

\item We implemented the proposed model checking technique and
reported verification results.

\item We highlighted the motivations and contributions of the
paper.

\end{itemize}

\newpage

%%%%%%%%%%%%%%%%%%%%%%%%% REVIEWER 1 %%%%%%%%%%%%%%%%%%%%%%%%%
\begin{center}
  \textbf{Reviewer 1}
\end{center}

\vspace{0.4cm}\textbf{\underline{1. Reviewer}:} 
You should relate V(S) to PO(C) in equations 5 and 6.

\vspace{0.2cm}\textbf{\underline{Answer 1}: 
We modified the section and added the connection at the end of paragraph, 
however we had already connected them when we were describing the algorithm 
of our scenario in section 4.1}


\vspace{0.4cm}\textbf{\underline{2. Reviewer}:} 
please add some definitions such as the one for Grand coalition
to make the paper self-contained.

\vspace{0.2cm}\textbf{\underline{Answer 2}: 
Great suggestion, we did add more definitions for second scneraio too.}



\vspace{0.4cm}\textbf{\underline{3. Reviewer}:} 
Why you didn't leverage the foundations of repeated games instead of
coalitional games? Is there any strong reason?.

\vspace{0.2cm}\textbf{\underline{Answer 3}: 
The communities are models as cooperative groups and that is why cooperative game thoery methods provide us with great decision mechanisims for forming groups and coalitions which would benefit all the agents involved altogether. In repeated games, there should be a repeatable game between agents one to one, where decision cannot be optimal for all the agents involved. We belieave repeated game is appropriate in cases where there is a lot of competing one-to-one interaction between agents are happening, and actually we are working on repeated games and specificly repeated game with learning techniques such as q-learning to help distribute the ``task distribution job'' among all the web services. Where web services are having different confidence states, and based on their action history and state they are in, web services can directly opt for getting and performing tasks alone or cooperating which other agents. 
}



\vspace{0.4cm}\textbf{\underline{4. Reviewer}:} 
What do you mean by "the algorithm gets interrupted": is it based on
a logical condition? if yes, what is it?

\vspace{0.2cm}\textbf{\underline{Answer 4}: 
It means if there is time constraint, the algorithm can get interrupted any time, and yet have depth(n-1) results. Each depth will have $O(n^2)$ cases to check, so for example if in 5 second it reaches depth 10, and we interrupt the algorithm since we cannot wait any loner,  we at least have checked for all subsets of depth 9}


\vspace{0.4cm}\textbf{\underline{5. Reviewer}:} 
Please provide equations using lambda and epsilon and relate them
to your model that uses Rc.

\vspace{0.2cm}\textbf{\underline{Answer 5}: 
We added the new stability($Core$ condition} in case of having taxtation.}

\vspace{0.4cm}\textbf{\underline{5. Reviewer}:} 
For the epslion core method: the value of epsilon should be
varied to verify its impact in the experiments section.

\vspace{0.2cm}\textbf{\underline{Answer 5}: 
That is exactly what we are doing, we changes the explanation to make what we did more clear.}

\vspace{0.4cm}\textbf{\underline{6. Reviewer}:} 
The experiments are missing a comparison with [18] is missing.

\vspace{0.2cm}\textbf{\underline{Answer 6}: 
The expirement which was illustrated in figure[9] was comparision of our work with [18], for all the scenarios which we could make it comparable and fair with.}

\vspace{0.4cm}\textbf{\underline{7. Reviewer}:} 
Scalability evaluation is missing: what is the impact of increasing the number of services/communities on the system efficiency?.

\vspace{0.2cm}\textbf{\underline{Answer 7}: 
As mentioned in the paper, our algorithms run in $O(n)$ and $O(n^2)$ time as number of web services with a community (n) grows. Also with the real web service data provided in http://www.uoguelph.ca/\textasciitilde{}qmahmoud/qws/ the largest communities in our expirements, did not grow larger than 60 web services. However in cases where $R_C$ is very large, communities can grow
large. However there will be no problem community management, and task distrobution algorithms as the number of web services grows, I could set $R_C$ very large in simulation but since we do not evaluate run time of comunity formation and membership strategies decisions, it will not affect our results.
}

All other comments are directly addressed in paper.

%%%%%%%%%%%%%%%%%%%%%%%%% REVIEWER 2 %%%%%%%%%%%%%%%%%%%%%%%%%
\vspace{2cm}

\begin{center}
  \textbf{Reviewer 2}
\end{center}

\vspace{0.4cm}\textbf{\underline{1. Reviewer}:}  Though there�s a referral to [8] it�d be helpful to explain more the concept of �Community coordinators� web services, somewhere in the early parts of the paper. How�s it different from other (common) web services in a given community.

``community master'' and ``community coordinator'' � are these different?


\vspace{0.2cm}\textbf{\underline{Answer 1}: We changed the way how we described our architecture, we dropped the concept of community master web service and we defined a separate entity as community coordinator.}

\vspace{0.4cm} \textbf{\underline{2. Reviewer}}: p.10 ``our community receives 130 tasks from users'' what are the types of tasks. Are these simulated or real functions that need to performed by web services?

\vspace{0.2cm}\textbf{\underline{Answer 2}: The tasks are randomly generated, and since communities are composed of web services performing same type of operations, they have same functional requirement and are of same type. In our implementation, we let communities perform tasks, and based on community task queue delay and also the assigned web service quality metric such as reliability, throughput, latency, successability and other metrics, the tasks will be performed with different qualities which we have evaluated and illustrated the results as average QoS of tasks performed}

All other comments are directly addressed in paper.


%%%%%%%%%%%%%%%%%%%%%%%%% REVIEWER 3 %%%%%%%%%%%%%%%%%%%%%%%%%
\vspace{2cm}

\newpage

\begin{center}
  \textbf{Reviewer 3}
\end{center}

\vspace{0.4cm}\textbf{\underline{1. Reviewer}:}  It is not convincing that the formation of service community can be modeled as a cooperative gaming problem. First, the concept of service community used in this paper is different from the ones used in the existing work, such as in the referred papers. The general idea of service community is a homogeneous service group that conceptually include services providing similar functionality. Services in the same community are independent and autonomous. There is no central control mechanism for the community. On the other hand, the concept of service community of this paper is a group of services which are monitored and managed by a community manager. It is not clear how realistic or practical such a community is and what the real world examples are.


\vspace{0.2cm}\textbf{\underline{Answer 1}: We changed the way how we described our architecture, we dropped the concept of community master web service and we defined a separate entity as community coordinator.}



\vspace{0.4cm}\textbf{\underline{2. Reviewer}:}  The problem is modeled in a way so that the cooperative game theory can be applied, but not in a way that is consistent with the real world scenarios. The service parameters, such as its request rate and throughout rate are not in line with the general characteristics of web services, such as input and output parameters. It is not clear how the proposed service parameters are obtained and why they are important to determine a service's membership to a community, considering that they depend on not only the service's capacity, but also the amount and frequency of user requests, which can be random.


\vspace{0.2cm}\textbf{\underline{Answer 2}: We changed the way how we described our architecture, we dropped the concept of community master web service and we defined a separate entity as community coordinator.}


\vspace{0.4cm}\textbf{\underline{3. Reviewer}:}  The paper conducted a simulation instead of an experimental study using real world web services. This also suggests the difficulties of finding real world examples and application for the proposed idea.


\vspace{0.2cm}\textbf{\underline{Answer 3}: We changed the way how we described our architecture, we dropped the concept of community master web service and we defined a separate entity as community coordinator.}




\vspace{0.4cm}

\vspace{1cm}


Finally, we hope that we have answered all the questions raised by
the reviewers, which will make the paper accepted for publication
in Applied Intelligence Journal.














{
%\bibliographystyle{plain}
%\bibliography{omar}
}
\end{document}

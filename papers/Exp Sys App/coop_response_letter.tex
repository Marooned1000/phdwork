
%\documentclass[preprint,authoryear,12pt]{elsarticle}
\documentclass[times, 12pt,a4paper]{article}
\usepackage{graphicx,amssymb,amsthm,stmaryrd}%,booktabs}
\usepackage{times}
\usepackage{amsmath,url,algorithm,algorithmic,mathrsfs,makeidx}
\usepackage{graphicx}
\usepackage{epstopdf}
\usepackage{epsfig}
\usepackage{latex8}

%journal{Journal of Knowledge-Based Systems}



\begin{document}
%\begin{frontmatter}

\newtheorem{definition}{Definition}
\newtheorem{proposition}{Proposition}
\newtheorem{example}{Example}
\newtheorem{lemma}{Lemma}
\newtheorem{theorem}{Theorem}
\newtheorem{corollary} {Corollary}
\title{\underline{Response Letter}\\\vspace{0.5cm}To Compete or Cooperate? This is the Question in Communities of Autonomous Services}
\author{Ehsan Khosrowshahi Asl\\
e\_khosr@encs.concordia.ca\\
\and
Jamal Bentahar\\
bentahar@ciise.concordia.ca \\
\and
Rabeb Mizouni\\
rabeb.mizouni@kustar.ac.ae \\
\and
Babak Khosravifar\\
babak.khosravifar@mcgill.ca \\
\and
Hadi Otrok\\
hadi.otrok@kustar.ac.ae \\
}


\maketitle


First, we would like to thank and express our appreciation to the
reviewers for taking the time to carefully review our manuscript,
and for their insightful, valuable, and very useful comments. We
would like to thank the associate and chief editors as well for
their valuable comments and suggestions for improvements. We
carefully considered the comments and implemented all of them in
the revised version. In this letter, we will explain, point by
point, how the issues raised in the reviews are addressed and
answered. We hope our answers and revisions will satisfy the
reviewers and editors, which will make the paper accepted for
publication in Expert Systems With Applications Journal.\\

%\newpage

%%%%%%%%%%%%%%%%%%%%%%%%% REVIEWER 1 %%%%%%%%%%%%%%%%%%%%%%%%%
\begin{center}
  \textbf{Reviewer 1}
\end{center}


\textbf{\underline{Reviewer}:} The presented paper is very good and up to date in the state of the art. The paper addresses a framework in which the agents use decision mechanisms in order to compete or cooperate.


\vspace{0.5cm} \textbf{\underline{1. Reviewer}:} The abstract should be shorter and it lacks information on the proposed framework and the decision mechanism based on game theory which are not mentioned. The suggested modifications aim make the abstract and the conclusion more similar.

\vspace{0.2cm}\textbf{\underline{Answer 1}: Thanks for the
comment. We made the abstract shorter and added the information about our decision making process which based on game-theoretic best response technique in the abstract.}

\vspace{0.5cm}\textbf{\underline{2. Reviewer}:}  In section 2.2 (system parameters), the authors should specify how many parameters are used before explaining them and using such parameters in the proposed mathematics. This should aid the reader in better understanding.

\vspace{0.2cm}\textbf{\underline{Answer 2}: We added a short summary about the system parameters in section 2.2. We have five system parameters and the other variables are derived or calculated based on these five parameters which are listed in Table 1. The title for Table 1. which was ``List of proposed system parameters'' was not very appropriate and we apologize for the confusion that it has caused. We changed its title to ``List of abbreviations''}


\vspace{0.5cm}\textbf{\underline{3. Reviewer}:} The text consists on very long paragraphs, mainly section 4 (experimental results), which makes the reading difficult.

\vspace{0.2cm}\textbf{\underline{Answer 3}: In the revised
version, all the needed concepts are defined. Grand coalition is
defined as follows: In cooperative game theory, grand coalition is
a coalition that contains a maximum number of players as long as
they can maintain a stable (in terms of membership) and beneficial
(in terms of utility) group for all the players (Section 4.1,
paragraph 3).}


\vspace{0.5cm}\textbf{\underline{4. Reviewer}:} Still in section 4, the simulation application should be better specified, adding information on used machine, architecture, most important used services, etc.

\vspace{0.2cm}\textbf{\underline{Answer 4}: The main objective of
this work is to optimize the gain of web service communities as a
whole and to ensure the creation of stable communities in terms of
joining and leaving activities.}



\vspace{0.5cm}\textbf{\underline{5. Reviewer}:} Figure 3 is misplaced.

\vspace{0.2cm}\textbf{\underline{Answer 5}: The interruption we
are mentioning can be either caused by time/memory constraints
when the number of subsets is big enough (timeout) or forced
manually by the system designer if for example the already
obtained results are satisfactory. In fact, if the algorithm gets
interrupted while verifying $depth(n)$, it will have $depth(n-1)$
already verified. Each depth will have $O(n^2)$ cases to check, so
for example if it reaches depth 10, and the algorithm gets
interrupted (manually or because of system resources), we at least
have checked for all the subsets of depth 9. Clarification has
been added in the revised manuscript at Section 4.1, last
paragraph.}


\vspace{0.5cm}\textbf{\underline{6. Reviewer}:} 6) There are some minor format errors that should be addressed.

\vspace{0.2cm}\textbf{\underline{Answer 6}: Tha
}

%\vspace{0.2cm}\textbf{\underline{Answer 9}: We 
%\url{http://www.uoguelph.ca/\textasciitilde{}qmahmoud/qws/}, in}


%%%%%%%%%%%%%%%%%%%%%%%%% REVIEWER 2 %%%%%%%%%%%%%%%%%%%%%%%%%
\vspace{2cm}

\newpage

\begin{center}
  \textbf{Reviewer 2}
\end{center}


\textbf{\underline{Reviewer}:} The paper perfectly fits within the scope of Expert Systems. This manuscript investigates the development of communities of autonomous services, formalizing the question "To compete or to cooperate?" between them. The authors do so by theorising and conducting a study through a dataset of 2,507 real services functioning on the web.


\vspace{0.5cm}\textbf{\underline{1. Reviewer}:}  Abstract: In my opinion, it is not necessary to talk about the sample used.

\vspace{0.2cm}\textbf{\underline{Answer 1}: First}

\vspace{0.5cm} \textbf{\underline{2. Reviewer}}:  Introduction: The first phrase ``Services are software systems...'' should include which kind of services are being analysed: intelligent, autonomous,... The definition used does not correspond to ``Services'' as a whole.

\vspace{0.2cm}\textbf{\underline{Answer 2}: We agree, we rephrased
the sentence as follows:  If the member $i$ is added to the set
$S$, the contribution of this set to the coalition is $v(S \cup
\left\{i\right\}) - v(S)$.}
\vspace{0.5cm} \textbf{\underline{3. Reviewer}}: Related work: The authors state in page 21: ``The interesting idea is to consider the benefits under four categories: innovation and learning,...''. If this is an interesting idea, it should be further explained. I agree with the authors, these benefits are very interesting, please provide further details.

\vspace{0.2cm}\textbf{\underline{Answer 3}: In the previous
version, the tasks were abstract. In the revised version, the
tasks are real tasks about flight booking that need to be.}


\vspace{0.5cm} \textbf{\underline{4. Reviewer}}:  Conclusion: The paper identifies clearly the implications for practice and further research. I consider the service consumer role very interesting for future research.

\vspace{0.2cm}\textbf{\underline{Answer 4}: We mean deviating from
the coalition. This has been fixed in the revised version.}



\vspace{0.2cm}\textbf{\underline{Answer 1}: As argued in our
Answer 2 to Reviewer 1, we are advocating the idea of community as
a way to group web services in order to increase the total gain of
all the members where each web service gain is more than or equal
to what this web service would obtain being outside the community.
This is because the community can provide wider visibility, better
market share, higher reputation, and better way of sharing
resources. Community for us is an infrastructure and being part of
it should benefit the web service, but also the community as a
whole. As argued in [Maamar et al., 2011]\footnote{Z. Maamar, P.
Thiran, and J. Bentahar (2011). Web services communities: from
intra-community Coopetition to inter-community Competition. In
E-Business Application for Product Development and Competitive
Growth: Emerging Technologies, In Lee editor, pp. 333-343, IGI
Global.}, inside a community, web services can either compete or
cooperate. In this paper, we stress the intra-community
cooperation between web services for a better inter-community
competition. The main idea is that to form a stable web service
community and to have fair distribution of the total gain based on
the contribution of each individual, cooperative game theory,
particularly coalition game theory, is very suitable to formalize
such a problem. Cooperation between competitors is a well
established practice in economical theories (see for instance
[Chetty and Wilson 2003]\footnote{S. K. Chetty and H. I. M. Wilson
(2003). Collaborating with competitors to acquire resources.
International Business Review, vol. 12. pp. 61-81.} and [Bucklin
and Sengupta 1993]\footnote{L. P. Bucklin and S. Sengupta (1993).
Organizing successful co-marketing alliances. Journal of
Marketing. vol. 57, pp. 32-46.}). This happens in real business
world scenarios. As an example, let us consider major carrier
airlines providing ticket selling services. Those airlines can be
seen as competing agents, each one tries to maximize its utility
by gaining more market share. However, nowadays those companies
are usually part of different coalitions (alliances) such as Star
Alliance\footnote{\url{www.staralliance.com}} (28 airline members
including Scandinavian Airlines, Thai Airways International, Air
Canada, Lufthansa, and United Airlines),
SkyTeam\footnote{\url{www.skyteam.com}} (19 members, among them
Aerom\'{e}xico, Air France, Delta Air Lines, Continental Airlines,
KLM, and Northwest Airlines), and Oneworld\footnote{\url{
www.oneworld.com}} (13 members including Air Berlin, American
Airlines, British Airways, Cathay Pacific, Iberia, Japan Airlines,
Malaysia Airlines, and Qatar Airways). By sharing resources and
market share and having the option of service multi-hop air
tickets combined from different airlines, each airline member of a
coalition can increase its gain by collaborating. The question is
when this collaboration would be beneficial for all the agents
involved? For example the three alliances would not gain anything
by collaborating and sharing resources with airlines of low
reputation and quality. They will lose their market and would
rather stay away from this collaboration. Therefore, no stable
community can be formed between well established and poor
airlines. Another example of cooperation among competitors is the
arrangement between PSA Peugeot Citro\"{e}n and Toyota to create
Toyota Peugeot Citro\"{e}n Automobile
(TPCA)\footnote{\url{www.tpca.cz/en/}} in order to share
components for a new city car where the different companies save
money on shared costs while remaining competitive in other areas.
Our solution of cooperation between competitor web services within
coalitions follows these models. We updated the paper to reflect
this important discussion by adding a new part, called
motivations, to the introduction (Section 1).}


\end{document}

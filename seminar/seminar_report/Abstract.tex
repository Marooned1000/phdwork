\begin{center}
{\LARGE\textbf{Abstract}}
\end{center}

In the last few years, communities of services have been studied in a certain numbers of proposals as virtual pockets of similar expertise. The motivation is to provide these services with high chance of discovery through better visibility, and to enhance their capabilities when it comes to provide requested functionalities. There are a number of proposed mechanisms and models on aggregating web services and making them cooperate within their communities. However, forming optimal and stable communities as coalitions to maximize individual and group efficiency and income for all the involved parties has not been addressed yet. Also, in the proposed frameworks of these communities, a common assumption is that residing services, which are supposed to be autonomous and intelligent, are competing over received requests, but also exhibit cooperative behaviors, for instance in terms of substituting each other.

In this report, we identify the problem of defining efficient algorithms for coalition formation mechanisms within communities and propose preliminary results using cooperative game-theoretic techniques. We propose a mechanism for community membership requests and selections of web services in the scenarios where there is interaction between one community and many web services and scenarios where web services can join multiple established communities. The ultimate objective is to develop a mechanism for web services to form stable groups allowing them to maximize their efficiency and generate near-optimal (welfare-maximizing) communities. The theoretical and extensive simulation results show that our algorithms provide web services and community owners, in real-world like environments, with applicable and near-optimal decision making mechanisms. We also propose a decision mechanism based on learning methods for services within communities to effectively choose the tasks to perform based on their capabilities and the competition between other community members.
























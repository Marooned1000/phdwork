\begin{center}
{\LARGE\textbf{Abstract}}
\end{center}

Web services are loosely-coupled business applications willing to
cooperate in distributed settings within different groups called
communities. Communities aim to provide better visibility,
efficiency, market share and total payoff. There are a number of
proposed mechanisms and models on aggregating web services and
making them cooperate within their communities. However, forming
optimal and stable communities as coalitions to maximize
individual and group efficiency and income for all the involved
parties has not been addressed yet.

In this proposal, we identify the problem of defining efficient
algorithms for coalition formation mechanisms within communities
and propose preliminary results using cooperative game-theoretic
techniques. We propose a mechanism for community membership
requests and selections of web services in the scenarios where
there is interaction between one community and many web services
and scenarios where web services can join multiple established
communities. The ultimate objective is to develop a mechanism for
web services to form stable groups allowing them to maximize their
efficiency and generate near-optimal (welfare-maximizing)
communities. The theoretical and extensive simulation results show
that our algorithms provide web services and community owners, in
real-world like environments, with applicable and near-optimal
decision making mechanisms.

As remaining work in this thesis, we propose to apply different
cooperative game techniques such as nucleus, kernel and bargaining
solution concepts and theoretically analyze those solution
concepts based on functions representing payoff and quality
metrics of the communities. We also aim to apply reinforcement
learning techniques for individual web service strategic decision
making process.























\setcounter{chapter}{3}
\chapter{Conclusion and Future Work}\label{sec:conclusionfuturework}

In this report, we proposed a cooperative game theory based model for the aggregation of web services within communities.
The goal of our services is to maximize efficiency by collaborating and forming stable
coalitions. Our method considers stability and fairness for all
web services within a community and offers an applicable mechanism
for membership requests and selection of web services. The
ultimate goal is to increase revenue by improving user
satisfaction, which comes from the ability to perform more tasks
with high quality. Simulation results show that our approximation
algorithms are polynomial in complexity and provide web services
and community owners with applicable and near-optimal decision
making mechanisms.

%As future work, we would like to perform more analytical and
%theoretical analysis on the convexity condition and also minimal $\epsilon$ values in \emph{$\epsilon$-core} solution concepts based on the
%characteristic function in web service applications. From web service perspective, the
%work can be extended to consider web service compositions where a
%group of web services having different set of skills cooperate to
%perform composite tasks. Also bargaining theory from cooperating
%game theory concepts can be used to help web services resolve the
%instability and unfairness issues by side payments.

\section {Future Plan and Timeline}

\indent The future goals in this Ph.D. research work are:

\begin{itemize}
\item Analyzing other cooperative solution concepts such as Kernel
and Nucleolus where payoff division is guaranteed to exist and may
have optimal results.

\item Analyzing the impact of Q-learning and reinforcement
learning on the performance of our model.

\item Developing a community membership algorithm technique for
our agents in more distributed and "incomplete information" settings.

%\item Developing an open source, Java based Tool, with UI for
%solving Core and Shapely solution concepts based on different
%input valuation functions.
\end{itemize}


\begin{center}
    \begin{tabular}{ | l | l | p{5cm} |}
    \hline
    Term & Activity & Publication \\ \hline
    Winter 2014 & 1 & 1 \\ \hline
    Summer 2014 & 1 & 1 \\ \hline
    Fall 2014 & 1 & 1 \\ \hline
    Winter 2015 & 1 & 1 \\
    \hline
    \end{tabular}
\end{center}


\subsection {Publications }

    \begin{enumerate}
        \item E. Khosrowshahi Asl, J. Bentahar, H. Otrok, R. Mizouni, \emph{``Efficient Community Formation for Web Services.''} Accepted at IEEE Transactions on Services Computing, Feb. 2014, IEEE. (Impact factor: 2.46)
        \item E. Khosrowshahi Asl, J. Bentahar, R. Mizouni, B. Khosravifar, H. Otrok, \emph{``To Compete or to Cooperate? This is the Question in Communities of Autonomous Services.''} In the Journal of Expert Systems and Applications, Feb. 2014, Elsevier. (Impact factor: 1.854, 5-year:  2.339)
        \item Published a conference paper in 10th IEEE International Conference on Services Computing; \emph{``Efficient Coalition Formation for Web Services''}, 2013
        \item Published a conference paper to the 10th International Conference on Service Oriented Computing; \emph{``Analyzing Coopetition Strategies of Services within Communities''}, 2012
    \end{enumerate}


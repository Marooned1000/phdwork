\setcounter{chapter}{1}
%*******************************START Background **********************************

\chapter{Background and Relevant Literature}\label{sec:MAS}



    \section{Community of Web Services}\label{sec:CommunityWS}


        \subsection{Web Services}\label{sec:BRWS}


        \subsection{Definition}\label{sec:CWSDefinition}

        \subsection{Architecture}\label{sec:CWSArchitecture}

        \subsection{Operations}\label{sec:CWSOperations}

            \subsubsection{Community Development}\label{sec:CWSCommunityDev}

            \subsubsection{Web Service Attraction and Retention}\label{sec:CWSAttraction}

            \subsubsection{Web Service Selection}\label{sec:CWSSelection}
            
        \subsection{Related Work}\label{sec:BRRelatedWork}


One of the first contributions in which web service community was introduced was in \cite{Zeng:2003:QDW:775152.775211}. They defined web service community as collection of web services providing same functionality however with different quality metrics. 

Most of the recent work on communities of services are either
user-centric and focus on user satisfaction
\cite{Chun02user-centricperformance} or system-centric and focus
on the whole system throughput, performance and utilization. There
are many contributions in distributed, grid, cluster and cloud
services which are system-centric. However, in real world
environments and applications, both users and service providers
are self-interested agents, aiming to maximize their own profit.
In those environments, both parties (users and services) will
collaborate as long as they are getting more benefits and payoff.

In this direction, recently \cite{DBLP:conf/IEEEscc/LimTMB12,
DBLP:conf/IEEEscc/KhosravifarABT11, 10.1109/TSC.2012.12} proposed mechanisms to help
users and services to maximize their gain. A two-player
non-cooperative game between web services and community master was
introduced in \cite{DBLP:conf/IEEEscc/KhosravifarABT11}. In this
game-theoretic model, the strategies available to a web service
when facing a new community are requesting to join the community,
accepting the master's invitation to join the community, or
refusing the invitation to join. The set of strategies for
communities are inviting the web service or refusing the web
service's join request. Based on their capacity, market share and
reputation, the two players have different set of utilities over
the strategy profiles of the game. The main limits of this game
model are: 1) its consideration of only three quality parameters,
while the other factors are simply ignored; and 2) the
non-consideration of the web services already residing within the
community. The game is only between the community master and the
new web service, and the inputs from all the other members are
simply ignored. The consideration of those inputs is a significant
issue as existing web services can lose utility or payoff because
of the new member, which can results in an unhealthy and unstable
group. The problem comes from the fact that the existing members
should collaborate with the new web services, so probably their
performance as a group can suffer. Existing members may even
deviate and try to join other communities if they are unsatisfied.
Those considerations of forming stable and efficient coalitions
are the main contributions of our paper.

In \cite{DBLP:conf/IEEEscc/LimTMB12}, a 3-way satisfaction approach
for selecting web services has been proposed. In this approach,
the authors proposed a web service selection process that the
community masters can use. The approach considers the efficiency
of all the three involved parties, namely users, web services and
communities. In this work, it is shown how the gains of these
parties are coupled together using a linear optimization process.
However, the optimization problem in this solution tends to
optimize some parameters considering all web services regardless
of their efficiency and contribution to the community's welfare.
Moreover, there are no clear thresholds for accepting or rejecting
new web services. The solution of the optimization problem could,
for instance, suggest web services already residing within the
community to increase or decrease their capacity to cover up the
weakness of other parties in the system. However, a high
performing web service could deviate anytime it finds itself
unsatisfied within the community instead of adjusting its service
parameters.

In \cite{10.1109/TSC.2012.12}, a cooperative scheme among autonomous
web services based on coalitional game theory has been introduced. They have proposed an algorithm to
reach individually stable coalition partition for web services in order to
maximize their efficiency. The communities choose new web services on the promise
that it would benefit the community without decreasing any other web service's
income. In their model, the worth of community is evaluated with high emphasis on
availability metric and considering price and cost values only. The community structure is based on a coordination chain,
where a web service is assigned as a \emph{primary} web service and the community task destribution
method, will initially invoke the primary web service and only if the primary web service is unavailable
will invoke the next backup web services as they are ordered in the coordination chain. However in cooperative models, it is preferred to
have a real and active cooperative activity engaging all agents to perform the tasks more efficiently. Especially nowadays
with recent advancement in cloud and hardware infrastructures availability is becoming less of an issue. So the backup web services
in their model have a very low chance of getting jobs, especially the ones further in chain, which is huge waste of web services
capabilities.

    \section{Cooperative Game Theory in Multiagent Systems}\label{sec:CGTMS}

        \subsection{Cooperative Game Thoery}\label{sec:BRWS}

        \subsection{Motivating Example}\label{sec:CWSDefinition}
        
        One the closest examples of cooperative situations, for our service model is \emph{Single landowner and landless workers} example. 
        This example was introduced in \cite{GVK369342747} and represents the \emph{landowner} entity which is providing jobs and \emph{landless workers} who cannot do anything by themselves and need to work with a landowner to gain some income.
        
        Suppose that the land is owned by a single person, the \emph{landowner}. We refer to the other people as \emph{workers}. In this case we have the following game model:
        
        \begin{itemize}
        \item \emph{Players} The landowner and the $m$ workers.
        \item \emph{Actions} A coalition consisting solely of workers has a single action in which no member receibes any output. The set of actions of a coalition $S$ consisting of the landowner and $k$ workers is the set of all $S$-allocations of the output $f(k+1)$ among the members of $S$.
        \item \emph{Preferences} Each player's preferences are represented by the amount of output he obtains.
        \end{itemize}

        \subsection{Cooperative Games in Service Oriented Computing}\label{sec:CWSArchitecture}

        \subsection{Representation and Complexity issues}\label{sec:CWSArchitecture}
        
        \subsection{Related Work}\label{sec:CGTMSRelatedWork}

Cooperative game theory provides a set of mathematical and optimization tools for multi-agent environments. These tools have been utilized in communication networks and service oriented computing literature, where nodes as rational agents try to reason strategically and maximise their benefit.





%*******************************End Background **********************************

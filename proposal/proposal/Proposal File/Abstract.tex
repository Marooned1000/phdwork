\begin{center}
{\LARGE\textbf{Abstract}}
\end{center}

Making decision in multiagent systems (MAS) environment is not an easy task, and it is still an open problem of this research field. It is a big
challenge for an agent in negotiation dialogue games to decide which move to play at each dialogue step, and which strategy will the agents employ
in order to convince his opponent and to achieve an agreement. In the other hand, what agent's type is suitable for which dialogue type is another 
challenge facing the agent's developers. In this research proposal, we tackle these problems by proposing a new framework for Argumentation-Based 
Negotiation (ABN) dialogue games. Our proposed work Consists of three phases; in phase one, we define a new set of uncertainty measures in negotiation 
dialogue games from an external agent's point of view. In particular, we introduce two types of uncertainty: Type I and Type II. Type I is about the 
uncertainty index of playing the right move. For this, we use Shannon entropy to measure: i) the uncertainty index of the agent that he is selecting 
the right move at each dialogue step; and ii) the uncertainty index of participating agents in the negotiation about the whole dialogue. This is done 
in two different ways; the first is by taking the average of the uncertainty index of all moves, and the second is by determining all possible dialogues 
and applying the general formula of Shannon entropy. Type II is about the uncertainty degree of the agent that the move will be accepted by the addressee.
In this context, we introduce a new classification for the arguments based on their certainty to be accepted by the addressee. In phase two, we extend
the uncertainty measures by developing a new uncertainty measurements such as nonspecificity, confusion, dissonance and discord using the theory of belief 
function (or Dempster-Shafer Theory (DST)) instead of the classical probability theory. Moreover, we combine and aggregate those measurements to finally 
measure the total uncertainty. In the third phase, we plane to investigate the negotiation strategies and concession strategies that agent can follow to 
achieve a better agreement. Analyze argumentation-based negotiation to evaluate agent strategies and analyze them from the optimization perspective is 
another target of our research. We belief that our proposed work presents some of the key techniques for reaching agreements in multi-agent environments.























\setcounter{chapter}{0}
%*******************************START INTRODUCTION **********************************
\chapter{Introduction}\label{sec:intro}
In this chapter we introduce the context of this research, which is about community of web services as autonomous agents and cooperative game theoretic solution concepts. We discuss litreture review and their shortcommings then present the motivations behind this work and the research questions. Also, we discuss the objective of this research and our preliminary contributions. Proposal organization is presented in the last section.

\section{Context of the research}\label{sec:context}


\section{Motivations and research questions}\label{sec:motivation}

Multiagent systems are widely used in everyday life, and to add more value to these systems in the field of software engineering,
they supposed to be measurable. Our motivation is to find a way to measure these systems from different aspects such as measuring
the dialogues, the performance of the participants in the dialogues, and the protocols governing the dialogues, etc. In order to evaluate
dialogues in multiagent systems, we define a new set of measurements from an external agent's point of view. Defining measures for the
participants in the dialogues is another motivation in this thesis. The aim behind developing such measurements is to help engineers and
developers of agent-based systems in evaluating these systems and their performances.\\

\indent When monitoring a dialogue between two or more agents, there are many question that should be answered. In this thesis, we are interested in answering the following questions:

\begin{itemize}
\item How much are agents certain about selecting a move at each dialogue step?
\item How much are agents certain about their dialogues?
\item How good are agents in the real dialogue (i.e. the effective dialogue)?
\item How far are agents from the right dialogue (i.e. the best dialogue given the knowledge bases of the participants)?
\end{itemize}
Answering these questions is undoubtedly complex. Therefore, we do not expect a comprehensive answer to all these questions.



\section{Research objectives and contributions}\label{sec:contribution}

The main objective of this thesis is to develop a new set of measurements for negotiation dialogue games to help in evaluating and
comparing different negotiation dialogues with different participants for the same topic. The importance of introducing measures for
negotiation dialogue games at each step of the dialogue such as measuring how much the agent is certain about its move is to help in developing intelligent multiagent systems (MAS) and to help evaluating different agents provided by different developers.\\
\indent Our research aims to ensure that the agents' certainty about their dialogue is fairly represented at each step during
the dialogue by making sure that the agent's certainty about selecting the right move is kept in mind and considered as a property
of multiagent systems.\\
\indent The main contribution of this thesis is the proposition of a new set of measures for dialogue games from an external agent's point of view.
In particular


\section{Research outline}\label{sec:outline}
The rest of the paper is organized as follows: In Section~\ref{sec:Argumentation}






%****************************************** END INTRODUCTION****************************************

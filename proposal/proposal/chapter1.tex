\setcounter{chapter}{0}
%*******************************START INTRODUCTION **********************************
\chapter{Introduction}\label{sec:intro}
In this chapter, we introduce the context of this research, which
is about communities of web services abstracted as autonomous
agents. Those agent-based web services use cooperative game
theoretic solution concepts for decision making. We discuss the
motivations of this work and briefly review the literature to
identify the problems we aim to solve in this thesis. Moreover, we
discuss our objectives and preliminary contributions.

\section{Context of Research}\label{sec:motivation}

Over the past years, online services have become part of many
scalable business applications. The increasing reliance on
web-based applications has significantly influenced the way web
services are engineered. Web services provide a set of stateless
software functions accessible at a network address over the web.
The recent developments are shifting web services from passive and
individual components to autonomous and group-based components
where interaction, composition, and cooperation are the key
challenges \cite{ICWS2011-1,SCC2011-1}. The main objective is to
achieve a seamless integration of business processes, applications
and web services. Delivering high quality services considering the
dynamic and unpredictable nature of the Internet is still a very
critical and challenging issue.

Typically, web services are business applications deployed as
autonomous and interoperable agents \cite{Alescio}. In fact, the
W3C consortium defines a web service as ``an abstract notion that
must be implemented by a concrete agent''. However, the web is
stocked with agent-based services that offer similar business
functionalities, which leads to service consumers having
difficulties in choosing the most appropriate agents to interact
with.

The need for highly available and responsive services has called
for grouping and collaborative mechanisms of loosely-coupled web
services, particularly in business settings. The idea of grouping
web services within communities and the way those communities are
engineered so that web services can better collaborate have been
proposed and investigated in
\cite{DBLP:journals/ijebr/MaamarSTBB09,DBLP:journals/internet/BenatallahSD03,Rosario:2008:PQS:1512146.1512290}.
Communities are virtual groups of web services having similar
functionalities \cite{Zeng:2003:QDW:775152.775211,
Paik:2005:TSS:2229263.2230038,Medjahed05adynamic,10.1109/ARES.2008.7},
but probably different non-functional quality attributes, which
form the QoS parameters. When communities are used, users send
their requests to the masters of those communities, which are
responsible of managing the communities, forwarding the requests
to the suitable member web services and checking the credentials
of those members. Communities aim to provide higher service
availability and performance than what individual web services can
provide. The high availability of services and the community
resilience to failure are guaranteed since web services can
cooperate and replace each other within the same community and
since there is no single point of failure in the communities
architecture.

\section{Motivating Scenario}\label{sec:motexample}

In this section, we present a scenario and demonstrate why there
is need for communities of services. We first propose an example
of a real world scenario, focusing on user experience. There are a
plethora of options available to people in today's society,
including weather forecasting, ticketing services, map services,
local places guides and so on. Most mobile or web applications
cannot independently satisfy users requests and should rely on
different online services. The high competition within the
services industry requires applications to use reliable and high
quality online service providers.


If the user were to check a web site or run an application on her
mobile device upon having downtime, or having high response delay
or encountering any non-satisfying quality metric, she will
instantly remove the application, which is a huge business concern
for application providers. For example, if a user installs a
ticketing application on her mobile device and the application is
not using high quality service providers, the user would instantly
uninstall the application, which has an extremely negative impact
on the
visibility of the application. %Let us say you search "Montreal
%Weather" on a search engine and you click on a web site which is
%slow because it is using a bad weather web service as source of
%its raw data, the user will leave the web site and try another
%link from search results, which has a bad Search Engine
%Optimization (SEO) effect for the web site, and search engines
%would not show that web site on search results anymore, which is a
%business and revenue loss for the application.
Thus, end user satisfaction is the main goal for competitive
online providers. Communities of web service, by providing
services with higher quality, higher uptime and reliability for
end users, aim to reach this goal. To this end, community
management decisions should capitalize on important QoS parameters
while forming the community and during membership management.


High demand on online services has created a massive business
competition. For example, nowadays users are provided with
multiple choices of web services offering local places information
such as coffees, venues, and shops nearby a geographic position.
It is hard for new web services to find their customers and be
visible for end users amongst hundreds or thousands of other
available services, even if they provide a high quality of
service. Hence, the concept of communities of web services
provides them with the chance of joining a platform with an
established market share and reputation. However, it is crucial
for a community manager to consider many factors when inviting or
accepting new members. For example, if the market share is not big
enough, bringing new web services can cause revenue drop for the
already residing members. This may encourage other web services to
collude, leave, or join other communities, hurting the community
stability. This is an important issue which has not been addressed
previously in the relevant literature. On the other hand, if
communities bound the number of web services to ensure higher
revenue, availability and response time could be negatively
affected if some members encounter problems. This is because
alternative web services for substitution will be limited.  This
has also not been efficiently addressed in the related work.
Consequently, community formation algorithms satisfying some
desirable properties such as community stability and overall
revenue are yet to be defined considering end users, community
managers and service providers.


\section{Motivation and Research Questions}\label{sec:researchquestions}

Web service communities are dynamic by design
\cite{DBLP:journals/ijebr/MaamarSTBB09}. In these communities, web
services are modeled as intelligent autonomous agents, where they
can adopt a strategy maximizing their payoff at any time. A web
service can ask joining a community and has the right to leave it.
Community managers can invite or ask a web service to leave in
order to maximize the community profit. Users can simply stop
sending requests to a web service which is not providing
satisfactory services. Thus its important to consider all the
parties involved in the decision making process about the
community management. Most of the recent work on communities of
services are either user-centric and focus on user satisfaction
\cite{Chun02user-centricperformance} or system-centric and focus
on the whole system throughput, performance and utilization. There
are many contributions in distributed, grid, cluster and cloud
services which are system-centric. However, in real world
environments and applications, both users and service providers
are self-interested agents, aiming to maximize their own profit.
In those environments, both parties (users and services) will
collaborate as long as they are getting more benefits and payoff.
Our initial research question is: \emph{How can we model the
community of agent-based services in order to maximize the utility
of involved users, web services and community organizers?}


In order to address this problem, recently
\cite{DBLP:conf/IEEEscc/LimTMB12,
DBLP:conf/IEEEscc/KhosravifarABT11, 10.1109/TSC.2012.12} proposed
mechanisms to help users and services maximize their gain. A
two-player non-cooperative game between web services and community
master was introduced in
\cite{DBLP:conf/IEEEscc/KhosravifarABT11}. In
\cite{DBLP:conf/IEEEscc/LimTMB12}, a 3-way satisfaction approach
for selecting web services has been proposed. In this approach,
the authors proposed a web service selection process that the
community masters can use. The approach considers the efficiency
of all the three involved parties, namely users, web services and
communities. The issue with these solution concepts is that they
consider community as a whole and model it as one entity in their
formulations. A community master decides on behalf of all the
members using an aggregated function of parameters. This can hurt
the overall revenue for some individual web services, or even a
subset of web services. Those services can collude and form their
own community and gain more, instead of having to adjust and share
their resources with other members. Another important issue which
needs to be considered is the community stability. In community of
web services, the members and community organizers collaborate to
perform tasks. Having jointly completed a task and generated
revenue, they need to agree on some reasonable method of dividing
profits (or tasks) among themselves. This is a key issue for the
group stability still to be investigated. If the revenue sharing
mechanism is not fair enough for any subset of web services
working in the community, these agents, as profit maximizing
entities, would just deviate and make their own group. So an
important research question that we would like to address is:
\emph{How can we model fair and stable communities as coalitions
of agent-based web services?}

%The consideration of those inputs is a significant
%issue as existing web services can lose utility or payoff because
%of the new member, which can results in an unhealthy and unstable
%group. The problem comes from the fact that the existing members
%should collaborate with the new web services, so probably their
%performance as a group can suffer. Existing members may even
%deviate and try to join other communities if they are unsatisfied.
%Those considerations of forming stable and efficient coalitions
%are the main contributions of our paper.

%However, a high
%performing web service could deviate anytime it finds itself
%unsatisfied within the community instead of adjusting its service
%parameters.




In \cite{10.1109/TSC.2012.12}, a cooperative scheme among
autonomous web services based on coalitional game theory has been
introduced. The authors have proposed an algorithm to reach
individually stable coalition partition for web services in order
to maximize their efficiency. The communities choose new web
services on the promise that it would benefit the community
without decreasing any other web service's income. In their model,
the worth of community is evaluated with high emphasis on the
availability metric and considering price and cost values only.
The community structure is based on a coordination chain, where a
web service is assigned as a \emph{primary} web service and the
community task distribution method will initially invoke the
primary web service. Only if the primary web service is
unavailable, the next backup web services in the ordered
coordination chain will be invoked. However, in cooperative
models, it is preferred to have a real and active cooperative
activity engaging all agents to perform the tasks more
efficiently. Thus, the final research question we would like to
work on is: \emph{How can we model and analyze the cooperation
among the community members in realistic, applicable and practical
settings?}

% ooooooold
%\indent When monitoring a dialogue between two or more agents, there are many question that should be answered. In this thesis, we are interested in answering the following questions:
%\begin{itemize}
%\item How much are agents certain about selecting a move at each dialogue step?
%\item How much are agents certain about their dialogues?
%\item How good are agents in the real dialogue (i.e. the effective dialogue)?
%\item How far are agents from the right dialogue (i.e. the best dialogue given the knowledge bases of the participants)?
%\end{itemize}
%Answering these questions is undoubtedly complex. Therefore, we do not expect a comprehensive answer to all these questions.

\section{Research Objectives and Contributions}\label{sec:motexample}

In this research work, our first objective is to propose a
cooperative model as game for the aggregation of web services
within communities. The solution concepts of our cooperative game
seeks to find efficient ways of forming coalitions (teams) of web
services so that they can maximize their gain and payoff, and
distribute the gain in a fair way among all the member services.
Achieving Fairness when the gain is distributed among the
community members is the main factor to keep the coalition stable
as no web service will expect to gain better by deviating from the
community. In other words, the coalition is made efficient if all
the members are satisfied. We first propose a representation
function for communities of web services based on their QoS
attributes. By using this function, we can evaluate the $worth$ of
each community of web services. When facing new membership
requests, a typical community master checks whether the new
coalition having the old and new set of web services will keep the
community stable or not. The community master will reject the
membership requests if it finds out that the new coalition would
be unstable, preventing $any$ subset of web services from gaining
significantly more by deviating from the community and joining
other communities or forming new ones. The computation of
solutions for cooperative games is combinatorial in nature and
proven to be NP-complete \cite{Algorithmic}, making this
computation impractical in real world applications. However, using
the concepts of coalition stability, the second objective is to
investigate approximation algorithms running in polynomial time
providing web services and community masters with applicable and
near-optimal decision making mechanisms.

\indent To summarize, the main problem we aim to tackle in this
thesis is the formation of stable and efficient coalitions
maximizing web services and community revenue. The main objectives
are:
\begin{itemize}
\item To propose a cooperative model and analyze its solution
concepts in order to address the problem of optimizing coalition
formation for a stable community.

\item To reduce the complexity of computing the solution concepts
of the cooperative model tailored to the problem of communities of
agent-based web services in order to make these solutions
applicable in real world scenarios.

\item To analyze the effect of different membership and taxation
models that the master can apply to the members on the stability
of the community.

\item To investigate the impact of learning on individual and
group decision making within the cooperative model of the
community.

\item To validate the proposed methods by extensive simulations
and comparison with other similar proposals.
\end{itemize}




\section{Proposal Organization}\label{sec:outline}
The rest of the proposal is organized as follows: We present in
Chapter~\ref{sec:MAS} the background needed for our research along
with relevant related work. Chapter~\ref{sec:coalitionformationws}
provides the problem statement and presents our solution model in
two different scenarios with some preliminary results. Finally, in
Chapter~\ref{sec:conclusionfuturework}, we present our conclusion,
future plan, and the timetable of our research.

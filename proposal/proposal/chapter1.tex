\setcounter{chapter}{0}
%*******************************START INTRODUCTION **********************************
\chapter{Introduction}\label{sec:intro}
In this chapter we introduce the context of this research, which is about community of web services as autonomous agents and cooperative game theoretic solution concepts. We discuss literature review and their shortcomings then present the motivations behind this work. Also, we discuss our preliminary contributions.

\section{Context of the research}\label{sec:context}


\section{Motivations and research questions}\label{sec:motivation}

Over the past years, online services have become part of many
scalable business applications. The increasing reliance on
web-based applications has significantly influenced the way web
services are engineered. Web services provide a set of stateless
software functions accessible at a network address over the web.
The recent developments are shifting web services from passive and
individual components to autonomous and group-based components
where interaction, composition, and cooperation are the key
challenges \cite{ICWS2011-1,SCC2011-1}. The main objective is to
achieve a seamless integration of business processes, applications
and web services. Delivering high quality services considering the
dynamic and unpredictable nature of the Internet is still a very
critical and challenging issue.

The need for highly available and responsive services has called
for grouping and collaborative mechanisms of loosely-coupled web
services, particularly in business settings. The idea of grouping
web services within communities and the way those communities are
engineered so that web services can better collaborate have been
proposed and investigated in
\cite{DBLP:journals/ijebr/MaamarSTBB09,DBLP:journals/internet/BenatallahSD03,Rosario:2008:PQS:1512146.1512290}.
Communities are virtual groups of web services having similar
functionalities \cite{Zeng:2003:QDW:775152.775211, Paik:2005:TSS:2229263.2230038,Medjahed05adynamic,10.1109/ARES.2008.7}, but probably different non-functional quality
attributes, which form the QoS parameters. When communities are
used, users send their requests to the masters of those
communities, which are responsible of managing the communities,
forwarding the requests to the suitable member web services and
checking the credentials of those members. Communities aim to
provide higher service availability and performance than what
individual web services can provide. The high availability of
services and the community resilience to failure are guaranteed
since web services can cooperate and replace each other within the
same community and since there is no single point of failure in
the communities architecture.

Most of the recent work on communities of services are either
user-centric and focus on user satisfaction
\cite{Chun02user-centricperformance} or system-centric and focus
on the whole system throughput, performance and utilization. There
are many contributions in distributed, grid, cluster and cloud
services which are system-centric. However, in real world
environments and applications, both users and service providers
are self-interested agents, aiming to maximize their own profit.
In those environments, both parties (users and services) will
collaborate as long as they are getting more benefits and payoff.

In this direction, recently \cite{DBLP:conf/IEEEscc/LimTMB12,
DBLP:conf/IEEEscc/KhosravifarABT11, 10.1109/TSC.2012.12} proposed mechanisms to help
users and services to maximize their gain. A two-player
non-cooperative game between web services and community master was
introduced in \cite{DBLP:conf/IEEEscc/KhosravifarABT11}. In this
game-theoretic model, the strategies available to a web service
when facing a new community are requesting to join the community,
accepting the master's invitation to join the community, or
refusing the invitation to join. The set of strategies for
communities are inviting the web service or refusing the web
service's join request. Based on their capacity, market share and
reputation, the two players have different set of utilities over
the strategy profiles of the game. The main limits of this game
model are: 1) its consideration of only three quality parameters,
while the other factors are simply ignored; and 2) the
non-consideration of the web services already residing within the
community. The game is only between the community master and the
new web service, and the inputs from all the other members are
simply ignored. The consideration of those inputs is a significant
issue as existing web services can lose utility or payoff because
of the new member, which can results in an unhealthy and unstable
group. The problem comes from the fact that the existing members
should collaborate with the new web services, so probably their
performance as a group can suffer. Existing members may even
deviate and try to join other communities if they are unsatisfied.
Those considerations of forming stable and efficient coalitions
are the main contributions of our paper.

In \cite{DBLP:conf/IEEEscc/LimTMB12}, a 3-way satisfaction approach
for selecting web services has been proposed. In this approach,
the authors proposed a web service selection process that the
community masters can use. The approach considers the efficiency
of all the three involved parties, namely users, web services and
communities. In this work, it is shown how the gains of these
parties are coupled together using a linear optimization process.
However, the optimization problem in this solution tends to
optimize some parameters considering all web services regardless
of their efficiency and contribution to the community's welfare.
Moreover, there are no clear thresholds for accepting or rejecting
new web services. The solution of the optimization problem could,
for instance, suggest web services already residing within the
community to increase or decrease their capacity to cover up the
weakness of other parties in the system. However, a high
performing web service could deviate anytime it finds itself
unsatisfied within the community instead of adjusting its service
parameters.

In \cite{10.1109/TSC.2012.12}, a cooperative scheme among autonomous
web services based on coalitional game theory has been introduced. They have proposed an algorithm to
reach individually stable coalition partition for web services in order to
maximize their efficiency. The communities choose new web services on the promise
that it would benefit the community without decreasing any other web service's
income. In their model, the worth of community is evaluated with high emphasis on
availability metric and considering price and cost values only. The community structure is based on a coordination chain,
where a web service is assigned as a \emph{primary} web service and the community task destribution
method, will initially invoke the primary web service and only if the primary web service is unavailable
will invoke the next backup web services as they are ordered in the coordination chain. However in cooperative models, it is preferred to
have a real and active cooperative activity engaging all agents to perform the tasks more efficiently. Especially nowadays
with recent advancement in cloud and hardware infrastructures availability is becoming less of an issue. So the backup web services
in their model have a very low chance of getting jobs, especially the ones further in chain, which is huge waste of web services
capabilities.


% ooooooold
%\indent When monitoring a dialogue between two or more agents, there are many question that should be answered. In this thesis, we are interested in answering the following questions:
%\begin{itemize}
%\item How much are agents certain about selecting a move at each dialogue step?
%\item How much are agents certain about their dialogues?
%\item How good are agents in the real dialogue (i.e. the effective dialogue)?
%\item How far are agents from the right dialogue (i.e. the best dialogue given the knowledge bases of the participants)?
%\end{itemize}
%Answering these questions is undoubtedly complex. Therefore, we do not expect a comprehensive answer to all these questions.



\section{Research objectives and contributions}\label{sec:contribution}

In this research work, we use game theory to
propose a cooperative game model for the aggregation of web
services within communities. The solution concepts of our
cooperative game seeks to find efficient ways of forming
coalitions (teams) of web services so that they can maximize their
gain and payoff, and distribute the gain in a fair way among all
the web services. Achieving Fairness when the gain is distributed
among the community members is the main factor to keep the
coalition stable as no web service will expect to gain better by
deviating for the community. In other words, the coalition is made
efficient if all the members are satisfied. We first propose a
representation function for communities of web services based on
their QoS attributes. By using this function, we can evaluate the
$worth$ of each community of web services. When facing new
membership requests, a typical community master checks whether the
new coalition having the old and new set of web services will keep
the community stable or not. The community master will reject the
membership requests if it finds out that the new coalition would
be unstable, preventing $any$ subset of web services from gaining
significantly more by deviating from the community and joining
other communities or forming new ones. The computation of
solutions for cooperative game theory problems is combinatorial in
nature and proven to be NP-complete \cite{Algorithmic}, making
this computation impractical in real world applications. However,
using the concepts of coalition stability, we propose
approximation algorithms running in polynomial time providing web
services and community masters with applicable and near-optimal
decision making mechanisms.


\section{Research outline}\label{sec:outline}
The rest of the paper is organized as follows: We present in Chapter~\ref{sec:MAS} the background 
needed for our research along with the relevant related work. Chapter 3 provides the problem statement and presents our solution model in two 
different scenarios with some primitive results. Finally in Chapter 4, we present our conclusion, future plan, and the timetable of our research.


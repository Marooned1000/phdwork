\setcounter{chapter}{3}

\chapter{Cooperative Game Solution Concepts}

This chapter was merged with chapter 2

\begin{definition}{\emph{[Preference Level].}} \label{preferenceLevel}
The preference level of a nonempty subset $\gamma$ of $\Gamma$ denoted by level $(\gamma)$ is the number of the highest numbered
layer which has a member in $\gamma$.
\end{definition}
\begin{example}
Let $\Gamma = \Gamma_1 \bigcup \Gamma_2 $ with $\Gamma_1$ = $\{ a, b\}$ and $\Gamma_2$ = $\{ c,d \}$ and $\gamma = \{a\}$ and
$\gamma\prime = \{a,d\}$. We have: level$(\gamma) = 1$ and level$(\gamma\prime) = 2$.
\end{example}


\begin{algorithm}{\emph{\textbf{Algorithm} 1.}} \label{algorithm1}\\
\\ Step 1:  $ W_{(H,h)/(H^{\prime},h^{\prime})}^{P_{Ag_1,Ag_2}} = 0 . $ \\
\\Step 2:  $(\forall x \in H),(\forall x^{\prime}  \in H^{\prime} ):
 ( pref(x,x^{\prime}) \in P_{Ag_1,Ag_2}^{pref})\Rightarrow W_{(H,h)/(H^{\prime},h^{\prime})}^{P_{Ag_1,Ag_2}} = W_{(H,h)/(H^{\prime},h^{\prime})}^{P_{Ag_1,Ag_2}}  +  1 $
\end{algorithm}


\begin{theorem}\label{Complexity1}
If arguments are represented in proportional definite Horn clauses, the arguments selection mechanism runs in polynomial time.
\end{theorem}

\noindent\emph{\textbf{Proof.}} \emph{It is known from Bentahar et al. \cite{BentaharIEEEIS2007}, that given a Horn knowledge base
$\Gamma$, a subset $H \subseteq \Gamma$, and a formula $h$; checking whether $(H, h)$ is an argument is polynomial. To decide
if an argument is irrelevant, we have to check if 1) $H \vdash \neg x$ for an $x \in CK$, which can be done in polynomial time
since $H$ is a definite Horn formula; or 2) there is a path from the root to $(H, h)$, which is a graph reachability problem, and
it is known by Jones \cite{Jones} that the problem is in NLOGSPACE. Since NLOGSPACE $\subseteq$ P, the problem can be
solved in polynomial time. To decide about the preference, we only need to compute the level of an argument from the level of a
subset of $\Gamma$, which is  a simple procedure that is obviously polynomial. Computing the favorite argument given two arguments
needs the computation of the arguments' weight, which is again a polynomial procedure as shown by Algorithm 1. To compere two given
arguments using the risk, we only need to compute the number of formulas in $H$ and check if they are part of different sets,
which is a polynomial procedure. Finally, the relevance ordering relation is simply based on comparing risks and favorites, which
are both polynomial, so we are done.}~~$\blacksquare$


        \begin{itemize}
            \item $x_i \geq 0$ for all $i \in N$, and
            \item $\sum_{i \in S} x_i \leq v(S)$
        \end{itemize}










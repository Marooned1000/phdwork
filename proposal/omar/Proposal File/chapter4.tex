\setcounter{chapter}{4}

\chapter{Conclusion and Future Work}

\section{Conclusion}\label{sec:conclusion}
In this paper, we proposed a new set of uncertainty measures for the agents in argumentation-based negotiation dialogues from an
external agent's point of view. Specifically, we introduced two types of uncertainty measures: 1) Type I, the uncertainty index of
playing the right move at each dialogue step; and 2) Type II, the uncertainty degree of the agent that the move will be accepted by
the addressee.
For uncertainty Type I, we used Shannon entropy to assess the agents' uncertainty/certainty about their moves in the negotiation
dialogues. We supposed that an external agent is monitoring the dialogue, and he wants to evaluate this dialogue in terms of the
agent's uncertainty/certainty about selecting the right move at each step. In fact, at each step, the agent is supposed to have
different choices, each choice is associated with a probability of being the right one, %which represents the risk of failure for this move to be accepted by the addressee.
this probability reflects the importance of information included in that move, where the higher the probability is, the
more certain the agent becomes (lower uncertainty). So we analyzed the fact that negotiating agents are rational, and they always
try to perform the actions that will result in the optimal outcome for themselves. We used Shannon entropy to measure: i) the
uncertainty/certainty index and the weighted uncertainty/certainty index of the agent that he is playing the right move at each
step during the dialogue; and ii) the uncertainty/certainty index and the weighted uncertainty/certainty index of both agents
participating in the dialogue about the whole dialogue. This was done in two different ways. The first is by taking the average of
the uncertainty index of all moves, and the second is by determining all possible dialogues and applying the general formula of Shannon entropy.

For uncertainty Type II, we formalized the probability association to the arguments and the uncertainty that the move will be
accepted by the addressee. In this context, we introduced a new classification of arguments based on the notion of risk of failure
and showed that this classification is compatible with the probability that the moves supported by those arguments will be
accepted by the address. An important result of this paper is that the selection and probability ordering mechanisms of arguments
are tractable as they can be performed in polynomial time if arguments are represented in propositional definite Horn logic.

In our proposed measures, the move with the higher certainty (lower uncertainty) index is considered as the best move. We analyzed
the fact that negotiating agents are rational, and they always try to perform the actions that will result in the optimal outcome
for themselves. We started our work with measuring the uncertainty/certainty index of each move at each dialogue step, and in order
to distinguish between two moves with the same certainty index, we assigned weight to each move, which reflects the importance of the
move. Then, we proceeded to the whole dialogue and we measured the uncertainty/certainty index for the dialogue in two different ways,
and we proved that the two ways give the same result. Also, assigning weight to the moves allowed us to compare two different dialogues.
We believe that such measures are very significant and helpful in evaluating the dialogue and the agent's strategies, especially when
they are making a decision and selecting the best moves to achieve an agreement (if one exists) in a timely manner.



\section{Future Work}\label{sec:future}
In this paper, we mainly focused on the classical (i.e., precise) probability theory, particularly the classical Shannonian
information theory to measure the uncertainty. As pointed out in \cite{Harmanec1999}, there are other approaches of measuring
uncertainty in the theories of imprecise probabilities, particularly the Dempster-Shafer theory (also known as the theory
of belief functions) \cite{Shafer76} and the possibility theory \cite{Dubois2001}. As future work, we will study key measurements
in these two theories such as \emph{nonspecificity}, \emph{confusion}, \emph{dissonance}, \emph{discord}, and \emph{strife}, and
we will adapt, define and integrate them in our framework. Combining and aggregating those measurements to finally measure the
total uncertainty will be investigated as well. The requirements of those new metrics as reported in \cite{Harmanec1999} will be
analyzed. Those requirements are: 1) generalization, meaning that the new uncertainty measures should generalize the uncertainty
measures already established in the classical probability theory; 2) subadditivity, meaning that when the problem is broken into
two orthogonal subproblems, the uncertainty of the original problem should be less than or equal to the sum of uncertainties of the
subproblems; and 3) additivity, meaning that under the assumption of no interaction and no dependency between the subproblems, the
uncertainty of the original problem is equal to the sum of uncertainties of the subproblems.

Another plan for future work is to extend the proposed metrics for other dialogue game types such as persuasion, deliberation,
inquiry and information seeking. We also plan to analyze argumentation-based dialogues to evaluate agent strategies and
analyze them from the optimization perspective. Analyzing the computational complexity of such optimization problems is another
direction for future work. Finally, we plan to analyze multi-party dialogues to which many agents can participate. Extending the
proposed metrics to this type of dialogues is not straightforward. For example, defining the rules of a multi-party negotiation is
much more complicated than two-party dialogues. In fact, multi-party dialogues cannot be simply reduced to many two-party dialogues.









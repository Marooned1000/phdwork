\setcounter{chapter}{2}
%*******************************START Background **********************************

\section{Introduction to Multiagent Systems}\label{sec:MAS}





\section{Types of Dialogue Games}\label{sec:DialogueGames}






\section{Negotiation Dialogue Games}\label{sec:Negotiation}
Negotiation is a form of interaction in which a group of self-interested agents with conflicting interests and a desire to cooperate attempt to 
reach  a mutual acceptable agreement on the division of scarce resources.



   \subsection{Negotiation Component}\label{sec:NComp}
   



   \subsection{Approaches to Automated Negotiation}\label{sec:Nappro}

Three types of approaches have been discussed in MAS literature as the authors argued in~\cite{JenningsPASC01}. These approaches are:
the \textit{game-theoretic} approach, the \textit{heuristic-based} approach, and the \textit{argumentation-based} negotiation (ABN)
approach.

The Game-theoretic approach is based on studying and developing strategic negotiation models based on game-theoretic precedents~\cite{Neumann44,JeffreyG94}.
The basic idea of this approach is to see the negotiation process as a game in which each participant tries to maximize his own utility. While this approach
is very promising in terms of results analysis, it suffers from some drawbacks as the authors argued in ~\cite{AmgoudV11}, due to the assumptions upon which
it is built. The most important ones are (i) the approach allows agents to exchange offers only, but not reasons or justifications, and (ii) the preference relation
$\succeq$ on offers is fixed during a negotiation for an agent. These assumptions are not realistic since in everyday life, other information than offers may be exchanged.
Moreover, it is very common that preferences on offers may change~\cite{AmgoudV11}.

The second approach is heuristic-based. This approach came to overcome some limitations of the game-theoretic approach (e.g., ~\cite{FatimaWJ04,JenningsPASC01}).
Heuristics are ad hoc rules that aim to achieve a good solution, but not necessarily an optimal one. That is why some strong assumptions made in the game-theoretic
approach such as the notion of rationality of agents as well as their resources are relaxed. Even though this approach came to cope with the game-theoretic approach
limitations, it does not solve the problem of the preference relation, i.e., the relation $\succeq$ remains the same during the negotiation~\cite{AmgoudV11}.

The third approach, which is our focus in this paper, is the argumentation-based negotiation approach (ABN). Plenty of research has been done on this approach as
witnessed by many publications, such as ~\cite{AmgoudPM00,Hunter04,Mbarki06,AmgoudDM07,AmgoudV11,EliseYP12}. The basic idea behind this approach is to allow the
participants of the negotiation dialogue not only to exchange offers, but also reasons and justifications that support these offers in order to mutually influence
their preference relation on the set of offers, and consequently to the out come of the dialogue. In~\cite{RahwanSD03}, the authors have introduced a particular
style of argument-based negotiation, namely interest-based negotiation (IBN), which is  a form of ABN in which agents explore and discuss their underlying interests.
%*******************************End Background **********************************
